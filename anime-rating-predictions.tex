% Options for packages loaded elsewhere
\PassOptionsToPackage{unicode}{hyperref}
\PassOptionsToPackage{hyphens}{url}
%
\documentclass[
]{article}
\usepackage{lmodern}
\usepackage{amssymb,amsmath}
\usepackage{ifxetex,ifluatex}
\ifnum 0\ifxetex 1\fi\ifluatex 1\fi=0 % if pdftex
  \usepackage[T1]{fontenc}
  \usepackage[utf8]{inputenc}
  \usepackage{textcomp} % provide euro and other symbols
\else % if luatex or xetex
  \usepackage{unicode-math}
  \defaultfontfeatures{Scale=MatchLowercase}
  \defaultfontfeatures[\rmfamily]{Ligatures=TeX,Scale=1}
\fi
% Use upquote if available, for straight quotes in verbatim environments
\IfFileExists{upquote.sty}{\usepackage{upquote}}{}
\IfFileExists{microtype.sty}{% use microtype if available
  \usepackage[]{microtype}
  \UseMicrotypeSet[protrusion]{basicmath} % disable protrusion for tt fonts
}{}
\makeatletter
\@ifundefined{KOMAClassName}{% if non-KOMA class
  \IfFileExists{parskip.sty}{%
    \usepackage{parskip}
  }{% else
    \setlength{\parindent}{0pt}
    \setlength{\parskip}{6pt plus 2pt minus 1pt}}
}{% if KOMA class
  \KOMAoptions{parskip=half}}
\makeatother
\usepackage{xcolor}
\IfFileExists{xurl.sty}{\usepackage{xurl}}{} % add URL line breaks if available
\IfFileExists{bookmark.sty}{\usepackage{bookmark}}{\usepackage{hyperref}}
\hypersetup{
  pdftitle={Anime Rating Predictions},
  pdfauthor={Clayton Ramsey},
  hidelinks,
  pdfcreator={LaTeX via pandoc}}
\urlstyle{same} % disable monospaced font for URLs
\usepackage[margin=1in]{geometry}
\usepackage{color}
\usepackage{fancyvrb}
\newcommand{\VerbBar}{|}
\newcommand{\VERB}{\Verb[commandchars=\\\{\}]}
\DefineVerbatimEnvironment{Highlighting}{Verbatim}{commandchars=\\\{\}}
% Add ',fontsize=\small' for more characters per line
\usepackage{framed}
\definecolor{shadecolor}{RGB}{248,248,248}
\newenvironment{Shaded}{\begin{snugshade}}{\end{snugshade}}
\newcommand{\AlertTok}[1]{\textcolor[rgb]{0.94,0.16,0.16}{#1}}
\newcommand{\AnnotationTok}[1]{\textcolor[rgb]{0.56,0.35,0.01}{\textbf{\textit{#1}}}}
\newcommand{\AttributeTok}[1]{\textcolor[rgb]{0.77,0.63,0.00}{#1}}
\newcommand{\BaseNTok}[1]{\textcolor[rgb]{0.00,0.00,0.81}{#1}}
\newcommand{\BuiltInTok}[1]{#1}
\newcommand{\CharTok}[1]{\textcolor[rgb]{0.31,0.60,0.02}{#1}}
\newcommand{\CommentTok}[1]{\textcolor[rgb]{0.56,0.35,0.01}{\textit{#1}}}
\newcommand{\CommentVarTok}[1]{\textcolor[rgb]{0.56,0.35,0.01}{\textbf{\textit{#1}}}}
\newcommand{\ConstantTok}[1]{\textcolor[rgb]{0.00,0.00,0.00}{#1}}
\newcommand{\ControlFlowTok}[1]{\textcolor[rgb]{0.13,0.29,0.53}{\textbf{#1}}}
\newcommand{\DataTypeTok}[1]{\textcolor[rgb]{0.13,0.29,0.53}{#1}}
\newcommand{\DecValTok}[1]{\textcolor[rgb]{0.00,0.00,0.81}{#1}}
\newcommand{\DocumentationTok}[1]{\textcolor[rgb]{0.56,0.35,0.01}{\textbf{\textit{#1}}}}
\newcommand{\ErrorTok}[1]{\textcolor[rgb]{0.64,0.00,0.00}{\textbf{#1}}}
\newcommand{\ExtensionTok}[1]{#1}
\newcommand{\FloatTok}[1]{\textcolor[rgb]{0.00,0.00,0.81}{#1}}
\newcommand{\FunctionTok}[1]{\textcolor[rgb]{0.00,0.00,0.00}{#1}}
\newcommand{\ImportTok}[1]{#1}
\newcommand{\InformationTok}[1]{\textcolor[rgb]{0.56,0.35,0.01}{\textbf{\textit{#1}}}}
\newcommand{\KeywordTok}[1]{\textcolor[rgb]{0.13,0.29,0.53}{\textbf{#1}}}
\newcommand{\NormalTok}[1]{#1}
\newcommand{\OperatorTok}[1]{\textcolor[rgb]{0.81,0.36,0.00}{\textbf{#1}}}
\newcommand{\OtherTok}[1]{\textcolor[rgb]{0.56,0.35,0.01}{#1}}
\newcommand{\PreprocessorTok}[1]{\textcolor[rgb]{0.56,0.35,0.01}{\textit{#1}}}
\newcommand{\RegionMarkerTok}[1]{#1}
\newcommand{\SpecialCharTok}[1]{\textcolor[rgb]{0.00,0.00,0.00}{#1}}
\newcommand{\SpecialStringTok}[1]{\textcolor[rgb]{0.31,0.60,0.02}{#1}}
\newcommand{\StringTok}[1]{\textcolor[rgb]{0.31,0.60,0.02}{#1}}
\newcommand{\VariableTok}[1]{\textcolor[rgb]{0.00,0.00,0.00}{#1}}
\newcommand{\VerbatimStringTok}[1]{\textcolor[rgb]{0.31,0.60,0.02}{#1}}
\newcommand{\WarningTok}[1]{\textcolor[rgb]{0.56,0.35,0.01}{\textbf{\textit{#1}}}}
\usepackage{longtable,booktabs}
% Correct order of tables after \paragraph or \subparagraph
\usepackage{etoolbox}
\makeatletter
\patchcmd\longtable{\par}{\if@noskipsec\mbox{}\fi\par}{}{}
\makeatother
% Allow footnotes in longtable head/foot
\IfFileExists{footnotehyper.sty}{\usepackage{footnotehyper}}{\usepackage{footnote}}
\makesavenoteenv{longtable}
\usepackage{graphicx,grffile}
\makeatletter
\def\maxwidth{\ifdim\Gin@nat@width>\linewidth\linewidth\else\Gin@nat@width\fi}
\def\maxheight{\ifdim\Gin@nat@height>\textheight\textheight\else\Gin@nat@height\fi}
\makeatother
% Scale images if necessary, so that they will not overflow the page
% margins by default, and it is still possible to overwrite the defaults
% using explicit options in \includegraphics[width, height, ...]{}
\setkeys{Gin}{width=\maxwidth,height=\maxheight,keepaspectratio}
% Set default figure placement to htbp
\makeatletter
\def\fps@figure{htbp}
\makeatother
\setlength{\emergencystretch}{3em} % prevent overfull lines
\providecommand{\tightlist}{%
  \setlength{\itemsep}{0pt}\setlength{\parskip}{0pt}}
\setcounter{secnumdepth}{-\maxdimen} % remove section numbering

\title{Anime Rating Predictions}
\author{Clayton Ramsey}
\date{November 17, 2020}

\begin{document}
\maketitle

\hypertarget{a-regression-model-for-predicting-anime-ratings}{%
\section{A Regression Model for Predicting Anime
Ratings}\label{a-regression-model-for-predicting-anime-ratings}}

\hypertarget{data-description}{%
\subsection{Data Description}\label{data-description}}

The data was retrieved from
\href{https://www.kaggle.com/CooperUnion/anime-recommendations-database}{Kaggle}.
It consists of user-generated ratings for 12294 anime productions such
as TV shows, movies, and OVAs.

I read the data and performed some cleaning tasks. First, I had to
convert the column representing number of episodes back to numeric. Some
cells contained the word ``unknown'', and this caused R to coerce the
column to character. I chose to keep only complete cases in order to
simplify the analysis.

This is the head of the data frame produced by this step:

\begin{longtable}[]{@{}ccccccc@{}}
\toprule
anime\_id & name & genre & type & episodes & rating &
members\tabularnewline
\midrule
\endhead
32281 & Kimi no Na wa. & Drama, Romance, School, Supernatural & Movie &
1 & 9.37 & 200630\tabularnewline
5114 & Fullmetal Alchemist: Brotherhood & Action, Adventure, Drama,
Fantasy, Magic, Military, Shounen & TV & 64 & 9.26 &
793665\tabularnewline
28977 & Gintama° & Action, Comedy, Historical, Parody, Samurai, Sci-Fi,
Shounen & TV & 51 & 9.25 & 114262\tabularnewline
9253 & Steins;Gate & Sci-Fi, Thriller & TV & 24 & 9.17 &
673572\tabularnewline
9969 & Gintama' & Action, Comedy, Historical, Parody, Samurai, Sci-Fi,
Shounen & TV & 51 & 9.16 & 151266\tabularnewline
32935 & Haikyuu!!: Karasuno Koukou VS Shiratorizawa Gakuen Koukou &
Comedy, Drama, School, Shounen, Sports & TV & 10 & 9.15 &
93351\tabularnewline
\bottomrule
\end{longtable}

As we can see the ``genre'' column consists of a comma-separated string
of genre tags. This isn't exactly suitable for an analysis, so I did tow
more processing steps to clean this data.

First, I separated the genres into discrete names. I wrote a function to
split the character strings and applied it to the ``genre'' column".

Then I made a comprehensive list of all listed genres.

To extract the nth genre of anime r I used
\texttt{x\ \textless{}-\ genres{[}{[}1{]}{]}{[}{[}r{]}{]}{[}n{]}}.

\begin{Shaded}
\begin{Highlighting}[]
\NormalTok{genre_split <-}\StringTok{ }\ControlFlowTok{function}\NormalTok{(x)\{}
  \KeywordTok{return}\NormalTok{(}\KeywordTok{str_split}\NormalTok{(x, }\StringTok{", "}\NormalTok{))}
\NormalTok{\}}
\NormalTok{genres <-}\StringTok{ }\KeywordTok{apply}\NormalTok{(}\KeywordTok{matrix}\NormalTok{(anime}\OperatorTok{$}\NormalTok{genre), }\DecValTok{2}\NormalTok{, genre_split)}
\KeywordTok{names}\NormalTok{(genres[[}\DecValTok{1}\NormalTok{]]) <-}\StringTok{ }\DecValTok{1}\OperatorTok{:}\KeywordTok{length}\NormalTok{(genres[[}\DecValTok{1}\NormalTok{]])}

\NormalTok{all_genres <-}\StringTok{ }\KeywordTok{character}\NormalTok{()}
\ControlFlowTok{for}\NormalTok{(i }\ControlFlowTok{in} \DecValTok{1}\OperatorTok{:}\KeywordTok{length}\NormalTok{(genres[[}\DecValTok{1}\NormalTok{]]))\{}
  \ControlFlowTok{for}\NormalTok{(j }\ControlFlowTok{in} \DecValTok{1}\OperatorTok{:}\KeywordTok{length}\NormalTok{(genres[[}\DecValTok{1}\NormalTok{]][[i]]))\{}
    \CommentTok{# Append the genre if it isn't there yet.}
    \ControlFlowTok{if}\NormalTok{(}\OperatorTok{!}\NormalTok{(genres[[}\DecValTok{1}\NormalTok{]][[i]][j] }\OperatorTok\StringTok{ }\NormalTok{all_genres))\{}
\NormalTok{      all_genres <-}\StringTok{ }\KeywordTok{append}\NormalTok{(all_genres, genres[[}\DecValTok{1}\NormalTok{]][[i]][j])}
\NormalTok{    \}}
\NormalTok{  \}}
\NormalTok{\}}
\end{Highlighting}
\end{Shaded}

Next, I created a new variable that records the number of genres each
anime is tagged for. The thinking here is that there may be some
connection between the ratings of the anime and its ``complexity'',
where the number of genres is a proxy for complexity. This variable is
bound to the data frame.

The \texttt{anime} data frame now looks like this:

\begin{longtable}[]{@{}cccccccc@{}}
\toprule
anime\_id & name & genre & type & episodes & rating & members &
genre\_tag\_number\tabularnewline
\midrule
\endhead
32281 & Kimi no Na wa. & Drama, Romance, School, Supernatural & Movie &
1 & 9.37 & 200630 & 4\tabularnewline
5114 & Fullmetal Alchemist: Brotherhood & Action, Adventure, Drama,
Fantasy, Magic, Military, Shounen & TV & 64 & 9.26 & 793665 &
7\tabularnewline
28977 & Gintama° & Action, Comedy, Historical, Parody, Samurai, Sci-Fi,
Shounen & TV & 51 & 9.25 & 114262 & 7\tabularnewline
9253 & Steins;Gate & Sci-Fi, Thriller & TV & 24 & 9.17 & 673572 &
2\tabularnewline
9969 & Gintama' & Action, Comedy, Historical, Parody, Samurai, Sci-Fi,
Shounen & TV & 51 & 9.16 & 151266 & 7\tabularnewline
32935 & Haikyuu!!: Karasuno Koukou VS Shiratorizawa Gakuen Koukou &
Comedy, Drama, School, Shounen, Sports & TV & 10 & 9.15 & 93351 &
5\tabularnewline
\bottomrule
\end{longtable}

This table contains the response variable \texttt{rating}, variables
suitable for predictors, and the variables \texttt{anime\_id} and
\texttt{name} which are not used as predictors.

The next step I wanted to make was to create a large data frame
containing factor variables representing the presence of the different
genres in each anime.

\begin{Shaded}
\begin{Highlighting}[]
\NormalTok{rating_by_genre <-}\StringTok{ }\KeywordTok{tibble}\NormalTok{(}\DataTypeTok{rating =}\NormalTok{ anime}\OperatorTok{$}\NormalTok{rating)}

\ControlFlowTok{for}\NormalTok{(i }\ControlFlowTok{in} \DecValTok{1}\OperatorTok{:}\KeywordTok{length}\NormalTok{(all_genres))\{}
  \CommentTok{#create a variable for that genre:}
\NormalTok{  x <-}\StringTok{ }\KeywordTok{integer}\NormalTok{(}\KeywordTok{length}\NormalTok{(anime}\OperatorTok{$}\NormalTok{genre))}
  \CommentTok{#populate the variable:}
  \ControlFlowTok{for}\NormalTok{ (j }\ControlFlowTok{in} \DecValTok{1}\OperatorTok{:}\KeywordTok{length}\NormalTok{(anime}\OperatorTok{$}\NormalTok{genre))\{}
\NormalTok{    x[j] <-}\StringTok{ }\NormalTok{all_genres[i] }\OperatorTok\StringTok{ }\NormalTok{genres[[}\DecValTok{1}\NormalTok{]][[j]]}
\NormalTok{  \}}
  \CommentTok{# Attach the newly created variable to the ratings:}
\NormalTok{  x <-}\StringTok{ }\NormalTok{x }\OperatorTok\StringTok{ }\KeywordTok{as_factor}\NormalTok{() }\OperatorTok\StringTok{ }\KeywordTok{as_tibble}\NormalTok{()}
  \KeywordTok{names}\NormalTok{(x) <-}\StringTok{ }\NormalTok{all_genres[i]}
\NormalTok{  rating_by_genre <-}\StringTok{ }\KeywordTok{cbind}\NormalTok{(rating_by_genre, x)}
\NormalTok{\}}

\KeywordTok{rm}\NormalTok{(i)}
\KeywordTok{rm}\NormalTok{(j)}
\end{Highlighting}
\end{Shaded}

The table looks like this. I'm only showing the first ten genres for
illustration purposes but there are actually forty-three of them.

\begin{longtable}[]{@{}cccccccccc@{}}
\toprule
rating & Drama & Romance & School & Supernatural & Action & Adventure &
Fantasy & Magic & Military\tabularnewline
\midrule
\endhead
9.37 & 1 & 1 & 1 & 1 & 0 & 0 & 0 & 0 & 0\tabularnewline
9.26 & 1 & 0 & 0 & 0 & 1 & 1 & 1 & 1 & 1\tabularnewline
9.25 & 0 & 0 & 0 & 0 & 1 & 0 & 0 & 0 & 0\tabularnewline
9.17 & 0 & 0 & 0 & 0 & 0 & 0 & 0 & 0 & 0\tabularnewline
9.16 & 0 & 0 & 0 & 0 & 1 & 0 & 0 & 0 & 0\tabularnewline
9.15 & 1 & 0 & 1 & 0 & 0 & 0 & 0 & 0 & 0\tabularnewline
\bottomrule
\end{longtable}

\hypertarget{visualizations}{%
\subsection{Visualizations}\label{visualizations}}

Looking at the correlations of the numeric variables:
\includegraphics{anime-rating-predictions_files/figure-latex/numeric-correlation-1.pdf}

The \texttt{members} variable is interesting here because it has some
sort of association with \texttt{rating} but it isn't really suitable
for a predictor variable. This is because the ratings and membership
have a mutually influential relationship. The variable \texttt{members}
is the ``number of community members that are in this anime's group'' on
myanimelist.net. It is likely that ``better'' anime has more of a
following, leading to more group members. So in some sense, the group
membership responds to ratings, not the other way around. But also, the
more people in the group could mean more exposure, gaining visibility
for that anime. This could cause people to view the anime and rate it.
And if there are more people in the group, it is likely to be better in
the first place, leading to better ratings.

Popularity and quality, represented by the variables \texttt{members}
and \texttt{rating} are associated and both suitable for response
variables. But it isn't wise to use one for a predictor of the other.
Not only that, the \texttt{members} variable isn't strictly about the
anime - it has to do with members of a website and so is at a remove
from the characteristics of the anime itself.

Here is a scatter plot of the \texttt{rating} variable against the
\texttt{members} variable. For anime with more members, the positive
relationship is more pronounced and the variance is smaller, possibly
representing a process of forming consensus.

\begin{verbatim}
## `geom_smooth()` using method = 'gam' and formula 'y ~ s(x, bs = "cs")'
\end{verbatim}

\includegraphics{anime-rating-predictions_files/figure-latex/members-rating-1.pdf}

The type appears to have some effect on the mean ratings:

\includegraphics{anime-rating-predictions_files/figure-latex/type-plot-1.pdf}

\begin{longtable}[]{@{}cc@{}}
\toprule
Type & Mean Rating\tabularnewline
\midrule
\endhead
Movie & 6.334486\tabularnewline
Music & 5.588665\tabularnewline
ONA & 5.640479\tabularnewline
OVA & 6.371087\tabularnewline
Special & 6.525454\tabularnewline
TV & 6.929487\tabularnewline
\bottomrule
\end{longtable}

The average rating for different numbers of genre tags shows a
potentially quadratic relationship:

\begin{verbatim}
## `geom_smooth()` using method = 'loess' and formula 'y ~ x'
\end{verbatim}

\includegraphics{anime-rating-predictions_files/figure-latex/genre-tags-1.pdf}

\end{document}
